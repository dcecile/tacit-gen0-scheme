\chapter{Welcome}

Welcome to the programming language Cottontail. It's a language
designed for practical and flexible game programming. Be assured that
you have my warmest welcome for taking the first steps into this land.

Before getting into the details, let me remind you of one irrevocable
guarantee. When you get stumped with no clear solution in sight, your
situation won't be derrided as marginal or wrongheaded. A solution
will be found; be it in documentation, in libraries, in tools, or in
language changes. If Cottontail can no longer be put to work solving
strange problems, it's time for it to retire. Scope and methodology be
damned, a solution will be found to your problem.

\pnote{Inspired by \cite{YES}.}

\section{Historic evolution}

By now you must be dying to know what exactly Cottontail is. If I tell
you that it's a JIT-compiled, functional, concatenative, polymorphic,
type-inferred language geared towards high performance and extensible
video games\ldots{} it might sound like a load of fluff. To really get
a feel for this language, I'll tell you how I got here.

It began with a strange problem that I was stumped with. I was merrily
coding up my game, with a bit of physics and graphics already done, I
was starting to consider level design. Level design in my case meant
custom events and logic, which led to thoughts of specialized,
interactive game objects. Common wisdom says that the solution here is
to integrate a scripting language. With a C++ codebase humming along,
I knew that throwing a scripting language like Lua into the mix would
start to put pressure on my performance goals, especially if
``customized logic'' included physics.

In this case, my core gameplay depended on physics, especially the fun
kind that requires lots of experimenting and tweaking in order to get
right. Quick feedback on my innovations is a perfect task for a
scripting language, right? As my mind contorted with the thought of
dividing my code into a static-performant part and a
flexible-inefficient part, I knew I was stumped. I didn't have the
insight or the guts to make that kind of a compromise.

I put the project down for a few years (unfortunately), and now with a
bit more insight into this craft of `software', I have the naïvity to
believe that a solution is within reach. I can imagine that it's
perfectly possible to write a game engine in a language as fast as
C++, and then edit a level, adding code in the same language, and
immediately test the changes without recompiling and without suffering
any performance penalties. Cottontail is the language that acheives
this goal, without compromising either performance, or interactivity.
It won't give your players low framerates, and it won't make you wait
even a second to try out new changes to your game.

Along the way of course, I've adopted a number of philosophies which
may help or hinder my progress towards this goal, but I stick with
them nonetheless. Most of it is simply about why I find programming
fun. You'll see evidence of these æsthetics as you progress through
this book:

\begin{itemize}

\item Type declarations are out. I've experienced a world of pain
trying to talk about types in C++ and Haskell. I've also experienced
the bliss in Python and Haskell of not having to mention types at all.

\item Generic programming is in. In Python and C++, working with
generic data structures can be a joy. In Cottontail, these benefits
are intended to extend far beyond mundane data structures.

\item Whitespace is significant. Python lets you declare blocks using
colons and indented lines---for a few of its keywords. Cottontail lets
you use significant whitespace wherever you want.

\item Functional style is permitted. Whipping up a few lambdas in
Haskell and stringing together pure combinators can go a long way
towards consise code. Even in C\#, I can enjoy full days without
having to use a for-loop.

\item Execution is concatenative. I'm a fan of ML's partial
application, but the calculus of operators (on par with Perl's) makes
my head spin. Concatenative programming (inspired by Forth) provides a
very nice solution to both function chaining and point-free
programming. As a bonus, gives a simple and consistent syntax.

\item Data is simple. I'll admit I was swayed by Douglas Hofstadter's
arguments in favour of modeling data using prototypes. And after using
Javascript's object literals, I'm no longer keen on designing class
heirarchies.

\item Garbage collection is avoided. Any veteran of C++ will be able
to explain to you the virtues of declaring objects on the stack and
passing them to other procedures as references, or better yet,
constant references. Cottontail makes these decisions on its own, but
the result should be very few cycles spend on garbage collection.

\item Errors are expected. There will be software bugs. Cottontail
tries to learn a few tricks from Erlang for the error-recovery
systems that it provides.

\item Small code is an objective. The combination of inferred types,
generic programming, and a functional style are intended to bring
dramatic simplifications and reductions in code. There's a reason why
Allan Kay is trying to build a 20K computer system that can be
printed in a 300 page book. There's also a reason why Steve Yegge
\cite{500K} finds his 500K game is already 5 times too large.

\item Scripting is the default. When I'm working on a game, I want
immediate feedback about what effect my changes have. This means
running the new code without closing or even reseting a running game.
Besides this productivity boost in this scenario, Cotttontail tries to
be a friendly scripting language that can be incorporated into any
level editor.

\item Performance is key. Everyone tries to hang on to their old
computers without upgrading, even as new software slows down their
systems. But I know how impressive it can be to see code push hardware
to its limit. Along this vein, Cottontail it aims to beat the
performance of C++ in game programming in memory usage and execution
speed.

\end{itemize}

\section{Learning Cottontail}

So, we start off in Cottontail with the REPL. The read-eval-print loop
is a staple of interactive scripting languages, and it will help you
find your feet without needing to open an IDE and without writing a
Makefile.

Once you have Cottontail installed, run the program named `Cotton'
(in Unix, you can run it from the terminal as `cotton').




